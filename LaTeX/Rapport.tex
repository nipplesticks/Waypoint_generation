\documentclass[a4paper]{article}
%\usepackage[swedish]{babel}
\usepackage[utf8]{inputenc}

%vid problem, testa någon av följande:
%\usepackage[cp1252]{inputenc} under Windows
%\usepackage[applemac]{inputenc} under MacOS 9 eller tidigare
%\usepackage[latin1]{inputenc} under MacOS X

\usepackage[T1]{fontenc}
\usepackage[pdftex]{graphicx}



%\LaTeX\ är kul. Den senaste stora versionen heter \LaTeX2e och släpptes 1994~\cite{lamport94}.
%\begin{equation}
%a^2+b^2=c^2
%\label{eq:Pythagoras}
%\end{equation}
%För Pythagoras ekvation, se Ekvation~\ref{eq:Pythagoras}.

\title{A* Pathfinding Acceleration with use of Auto-Generated Waypoints for Grid Traversal}
\author{Fredrik Olsson, Magnus Nyqvst}
\date{\today} 

\begin{document}
\pagenumbering{gobble}
\maketitle
\newpage
\thispagestyle{empty}
\paragraph{Abstract--}
Sammanfattar rapporten
Varför är vår rapport värd att läsa?
Syfte, metod
Viktiga resultat och slutsatser Nyckelord
“Tänk på att detta skall kunna läsas fristående”

\tableofcontents
\listoffigures
\newpage
\pagenumbering{arabic}
\twocolumn
\section{Introduction}
This is introduction lol \newline
Syfte \newline
Frågeställning \newline
Hypoteser \newline
Avgränsningar

\section{Background}
Stort spel \newline
Bakgrundsfakta (Definitioner som du använder dig av senare) \newline
Saker som läsaren behöver veta för att förstå din rapport \newline
Vad har gjorts tidigare?

\section{Related work}
Informationssökning \newline
Urval av litteratur - skriv inte om allt utan det viktigaste för din rapport

\section{Method}
Hur vi autogenerar (mycket referens till research articles) \newline
Vad har du gjort? \newline
Vilka metoder har du använt? \newline
Vad har du kommit fram till? \newline
Var noggrann och utförlig så att det går följa vad du gjort? \newline
Motivera om du antar något \newline
Diskutera begränsningar

\section{Result}
Bevisa att auto waypoints äger fett

Bilder kan roteras med hjälp av en dator, se exempel i Figur~\ref{fig:Logo}.

\begin{figure}[h]
\centering
\includegraphics[scale=0.9, angle=20]{bth_logo.jpg}
\includegraphics[scale=0.8, angle=40]{bth_logo.jpg}
\includegraphics[scale=0.7, angle=60]{bth_logo.jpg}
\caption{Roterade BTH-logotyper}
\label{fig:Logo}
\end{figure}

\section{Discussion}
Discuss the results

\section{References}
References to RA
\begin{thebibliography}{9}

\bibitem{lamport94}
  Leslie Lamport,
  \textit{\LaTeX: a document preparation system},
  Addison Wesley, Massachusetts,
  2nd edition,
  1994.

\end{thebibliography}

\section{Appendix and index}
Man vet aldrik


\end{document}
